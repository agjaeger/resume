
\resheading{School Projects}
\begin{itemize}
\item
	\ressubheading{SpectraVu Medical}{Vancouver, BC}{Engineering Physics Project Lab, APSC 479}{Sep. 2001 - Apr. 2002}
	\begin{itemize}
		\resitem{	Designed and implemented a digital video processing system for lung cancer imaging,}
		\resitem{Selected components (video DAC, ADC) and created schematics in OrCAD.}
		\resitem{Implemented image processing functions and data control blocks in VHDL using an Altera ACEK1K FPGA.  Learned VHDL and MAX+PlusII development tool on my own time.}
		%\resitem{Simulated the various blocks using MAX+PlusII simulation tool.}
	\end{itemize}

\item
	\ressubheading{Analog Circuit Design and MOSFET Device Design}{}{Semiconductor Devices Course, EECE 480}{Sep. 2001 - Apr. 2002}
	\begin{itemize}
		\resitem{Designed a high-frequency cascode amplifier, simulated it using HSPICE, and did layout using Cadence Virtuoso Layout software.  Manufactured on a Gennum GA911 chip.}
		\resitem{Designed and simulated a deep sub-micron (~70 nm channel) MOSFET using MEDICI.}
	\end{itemize}

\item
	\ressubheading{Low-cost Optoelectronic Localizer}{}{Engineering Physics Project Lab, APSC 459}{Sep. 2000 - Apr. 2001}
	\begin{itemize}
		\resitem{Worked on the LoCOL (Low-cost Optoelectronic Localizer) project in a team of three.}
		\resitem{Programmed a PIC microcontroller to control the timing of the three CCD cameras.}
		\resitem{Designed power supply and re-built electrical circuits for the CCD sensors, processors.}
		%\resitem{Helped build an enclosure for optics and electronics in the Student Machine Shop.}
	\end{itemize}

\item
	\ressubheading{Other Projects}{}{UBC and at home}{1999-2000}
	\begin{itemize}
		\resitem{Designed and debugged a digital voltmeter using a Motorola 68000 processor.}
		\resitem{Added features to the digital voltmeter including scrolling text, and a warning buzzer, which won 3rd place in the IEEE Voltmeter Competition.}
		\resitem{Constructed and debugged a digital clock on a PCB for PHYS 159.}
		\resitem{Built an AM short-wave radio at home, on a 2" $\times$ 2.5'' piece of breadboard.}
	\end{itemize}

\end{itemize}
